\introduction

Место прохождения практики – общество с ограниченной ответственностью «ВК Цифровые Технологии», отдел Research \& Development, команда Tarantool Platform Core. Период прохождения практики - с 1 июля 2024 года по 28 июля 2024 года.

В задачи прохождения практики входило:

\begin{enumerate}
    \item Разобраться в работе асинхронной репликации СУБД Tarantool;
    \item Изучить протокол подключения новых реплик к существующему репликасету Tarantool;
    \item Разработать дизайн-документ по фильтрации репликационного потока.
\end{enumerate}

Основная цель прохождения практики заключалась в разработке дизайн-документа по фильтрации репликационного потока в СУБД Tarantool для улучшения работы CDC. От заказчика (команда разработки CDC) к новой функциональности были предъявлены следующие требования:

\begin{enumerate}
    \item Фильтрация репликационного потока по именам спейсов. Это необходимо для того, чтобы только выбранные спейсы передавались репликам, что обеспечивает более гибкое управление репликацией и снижает нагрузку на систему;
    \item Возможность продолжения подключения реплики к репликасету с места предыдущей остановки. Это позволяет уменьшить время восстановления работы реплик CDC после сбоев;
    \item Сохранение данных для анонимных реплик. Это необходимо, чтобы исключить ошибки репликации, связанные с удалением ненужных с точки зрения репликасета данных механизмом очистки (GC).
\end{enumerate}

Каждое из вышеуказанных требований направлено на улучшение работы кластерной системы Tarantool, обеспечивая высокую отказоустойчивость, масштабируемость и эффективность обработки данных в распределенной среде. Разработанный документ является основой для последующей реализации предложенных решений в коде и их интеграции в существующую архитектуру Tarantool.
