\structure{ОСНОВНАЯ~ЧАСТЬ}

\section{Характеристика организации}

ВK Цифровые Технологии – подразделение VK, развивающее продукты и сервисы для цифрового бизнеса. В основе экосистемы решений VK Цифровые технологии лежит многолетний опыт развития интернет-сервисов и технологии на базе открытого кода. VK Цифровые Технологии предоставляет готовые сервисы для решения бизнес задач любой сложности, занимается заказной разработкой и управлением ИТ-инфраструктурой на аутсорсе \cite{VkTech}.

В портфеле VK цифровые Технологии — облачные сервисы VK Cloud Solutions, платформа in-memory вычислений Tarantool, платформа взаимодействия бизнеса и государства VK Tax Monitoring, а также линейка программных продуктов для управления персоналом, автоматизации производства и бизнес-процессов.

Tarantool как продукт появился 4 апреля 2016 года, когда Mail.ru Group (на данный момент известная как VK) сообщила о создании нового направления бизнеса, в рамках которого компания начала предоставлять корпоративным клиентам услуги в области хранения данных.

Изначально Tarantool применялся только в собственных проектах Mail.ru, в том числе в почтовом сервисе и облачном хранилище «Облако Mail.Ru». Затем компания превратила эту СУБД в продукт с открытым исходным кодом, который к началу апреля 2016 года внедрен рядом российских и международных компаний. В частности, Tarantool начал использоваться сервисом бесплатных объявлений Avito, социальной сетью знакомств Badoo и разработчиком систем информационной безопасности Wallarm.

На сегодняшний день Tarantool активно используется в банковской сфере (среди клиентов Tarantool можно выделить ВТБ, Альфа Банк, Банк Открытие и Газпромбанк) и для ретейла и e-commerce (Магнит, Wildberries, Ситилинк, X5Group) \cite{Tarantool}.
