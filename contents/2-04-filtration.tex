\section{Фильтрация репликационного потока}

Поcледним требованием заказчика была фильтрация репликационного потока. Это требование вытекает из необходимости снижения нагрузки на CDC реплики и заключается в частичной пересылки данных мастером.

Основные моменты:

\begin{itemize}
    \item Фильтрация репликации производится по именам спейсов. Фильтруется как процесс подключения реплик, так и процесс применения изменений. Клиент может подписаться на изменения спейсов, которые еще не были созданы.
    \item Запрошенные спейсы реплицируются вместе с их метаданным: \_space, \_index, \_truncate частично реплицируются, только данные, относящиеся к запрошенным спейсам, будут отправлены.
    \item При изменении имени или удаление спейса клиент получает информацию об этом последней записью. Последующие изменения (в случае переименования) не посылаются. Если будет создан спейс с именем, который запрошен клиентом, то клиент начинает получать данные из него.
\end{itemize}

Для обеспечения фильтрации в IPROTO\_FETCH\_SNAPSHOT и IPROTO\_SUBSCRIBE добавляется новая опция: IPROTO\_SPACE\_NAME\_FILTER, представляющая собой массив названий необходимых спейсов. Если опция не указана, полная репликация используется для обратной совместимости.

Управление подпиской на фильтрованный репликационный поток:

\begin{itemize}
    \item Добавление нового еще не существующего спейса к фильтру. Клиент может инициировать IPROTO\_SUBSCRIBE в любой момент, чтобы добавить новый спейс.
    \item Удаление спейса из фильтра. Делается таким же образом.
    \item Добавление уже существующего спейса в фильтр. Недопустимое изменение фильтра. На момент изменения фильтра клиент уже имеет какой-то vclock и мы продолжим посылать данные с этого vclock-a. Однако метаданные спейса будут скорее всего пропущены, так как спейс уже был создан до изменения фильтра.
\end{itemize}
