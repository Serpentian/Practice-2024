\newglossaryentry{id7}{
    name={Анонимная реплика},
    description={это тип реплики, которая подключается к ведущему узлу для получения данных, но не участвует в кворуме репликации и не может стать мастером репликасета.}
}

\newglossaryentry{id1}{
    name={База данных (БД)},
    description={это организованная коллекция данных, которая структурирована таким образом, чтобы данные можно было легко хранить, управлять, изменять и извлекать.},
}

\newglossaryentry{id3}{
    name={Захват изменения данных (Change Data Capture, CDC)},
    description={это программное обеспечение для отслеживания и записи изменений, происходящих в данных базы данных. Оно позволяет фиксировать все вставки, обновления и удаления записей в режиме реального времени или с минимальной задержкой, что позволяет синхронизировать данные между различными системами, вести аудит изменений, и создавать системы резервного копирования или аналитики.}
}

\newglossaryentry{id6}{
    name={Кластер},
    description={это совокупность нескольких репликасетов, каждый из которых чаще всего хранит разный набор данных.}
}

\newglossaryentry{id10}{
    name={Локальный Спейс (local space)},
    description={это спейс, данные из которого не реплицируются и остаются локальными для конкретного узла.}
}

\newglossaryentry{id5}{
    name={Репликасет (replicaset)},
    description={это группа узлов (инстансов), работающих в режиме репликации и объединенных для обеспечения отказоустойчивости и доступности данных.}
}

\newglossaryentry{id2}{
    name={Система управления базами данных (СУБД)},
    description={это программное обеспечение, предназначенное для создания, управления и обеспечения доступа к базам данных. Оно позволяет пользователям определять, создавать, изменять и управлять базой данных, а также обеспечивает взаимодействие между пользователями и базой данных через запросы и команды. Основные функции включают хранение, поиск, обновление и удаление данных, а также обеспечение целостности, безопасности и управления доступом к данным.}
}

\newglossaryentry{id4}{
    name={Спейс (space)},
    description={это основная логическая единица хранения данных Tarantool, аналогичная таблице в традиционных реляционных базах данных. Спейс содержит набор записей, каждая из которых называется кортежем (tuple). Структура спейса определяется схемой, которая включает количество и типы полей в кортежах, а также индексы для быстрого доступа к данным.}
}

\newglossaryentry{id8}{
    name={LSN},
    description={это монотонно возрастающий идентификатор записи.}
}

\newglossaryentry{id9}{
    name={Vclock},
    description={это массив LSN, идентификаторами в котором являются ID узлов. Vclock представляет собой набор логических счетчиков для каждого узла в кластере, позволяя определить, какие изменения были применены на конкретном узле и какие еще предстоит синхронизировать.}
}
